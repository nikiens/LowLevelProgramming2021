\documentclass[a4paper,12pt]{extarticle}
\usepackage[utf8x]{inputenc}
\usepackage[T1,T2A]{fontenc}
\usepackage[russian]{babel}
\usepackage{hyperref}
\usepackage{indentfirst}
\usepackage{listings}
\usepackage{color}
\usepackage{here}
\usepackage{array}
\usepackage{multirow}
\usepackage{graphicx}

\usepackage{caption}
\renewcommand{\lstlistingname}{Программа} % заголовок листингов кода

\bibliographystyle{ugost2008ls}

\usepackage{listings}
\lstset{ %
extendedchars=\true,
keepspaces=true,
language=C,						% choose the language of the code
basicstyle=\footnotesize,		% the size of the fonts that are used for the code
numbers=left,					% where to put the line-numbers
numberstyle=\footnotesize,		% the size of the fonts that are used for the line-numbers
stepnumber=1,					% the step between two line-numbers. If it is 1 each line will be numbered
numbersep=5pt,					% how far the line-numbers are from the code
backgroundcolor=\color{white},	% choose the background color. You must add \usepackage{color}
showspaces=false				% show spaces adding particular underscores
showstringspaces=false,			% underline spaces within strings
showtabs=false,					% show tabs within strings adding particular underscores
frame=single,           		% adds a frame around the code
tabsize=2,						% sets default tabsize to 2 spaces
captionpos=t,					% sets the caption-position to top
breaklines=true,				% sets automatic line breaking
breakatwhitespace=false,		% sets if automatic breaks should only happen at whitespace
escapeinside={\%*}{*)},			% if you want to add a comment within your code
postbreak=\raisebox{0ex}[0ex][0ex]{\ensuremath{\color{red}\hookrightarrow\space}},
texcl=true,
inputpath=listings,                     % директория с листингами
}

\usepackage[left=2cm,right=2cm,
top=2cm,bottom=2cm,bindingoffset=0cm]{geometry}

%% Нумерация картинок по секциям
\usepackage{chngcntr}
\counterwithin{figure}{section}
\counterwithin{table}{section}

%%Точки нумерации заголовков
\usepackage{titlesec}
\titlelabel{\thetitle.\quad}
\usepackage[dotinlabels]{titletoc}

%% Оформления подписи рисунка
\addto\captionsrussian{\renewcommand{\figurename}{Рисунок}}
\captionsetup[figure]{labelsep = period}

%% Подпись таблицы
\DeclareCaptionFormat{hfillstart}{\hfill#1#2#3\par}
\captionsetup[table]{format=hfillstart,labelsep=newline,justification=centering,skip=-10pt,textfont=bf}

%% Путь к каталогу с рисунками
\graphicspath{{fig/}}

%% Внесение titlepage в учёт счётчика страниц
\makeatletter
\renewenvironment{titlepage} {
 \thispagestyle{empty}
}
\makeatother


\begin{document}	% начало документа

% Титульная страница
\begin{titlepage}	% начало титульной страницы

	\begin{center}		% выравнивание по центру

		\large Санкт-Петербургский политехнический университет Петра Великого\\
		\large Институт компьютерных наук и технологий \\
		\large Высшая школа интеллектуальных систем и суперкомпьютерных технологий\\[6cm]
		% название института, затем отступ 6см
		
		\huge Низкоуровневое программирование\\[0.5cm] % название работы, затем отступ 0,5см
		\large Отчет по лабораторной работе №2\\[0.1cm]
		\large Программирование EDSAC\\[5cm]

	\end{center}


	\begin{flushright} % выравнивание по правому краю
		\begin{minipage}{0.25\textwidth} % врезка в половину ширины текста
			\begin{flushleft} % выровнять её содержимое по левому краю

				\large\textbf{Работу выполнил:}\\
				\large Аникин Д.А.\\
				\large {Группа:} 3530901/90004\\
				
				\large \textbf{Преподаватель:}\\
				\large Алексюк А.О.

			\end{flushleft}
		\end{minipage}
	\end{flushright}
	
	\vfill % заполнить всё доступное ниже пространство

	\begin{center}
	\large Санкт-Петербург\\
	\large \the\year % вывести дату
	\end{center} % закончить выравнивание по центру

\end{titlepage} % конец титульной страницы

\vfill % заполнить всё доступное ниже пространство


% Содержание
% Содержание
\renewcommand\contentsname{\centerline{Содержание}}
\setcounter{tocdepth}{1}
\tableofcontents
\newpage



\section{Цель работы}
Построить машину Тьюринга-Поста, вполняющую вычитание чисел в десятичном коде (уменьшаемое >= вычитаемому). Выполнить моделирование ее работы в одном из свободно доступных симуляторов. 

\section{Правила кодирования}

\subsection{Алфавит машины}
Алфавит машины состоит из следующих символов: ''-'' (разделение уменьшаемого и вычитаемого); Символы 1, 2, 3, 4,5, 6, 7, 8, 9,  используются для записи чисел в десятичном коде. Символы A, B, C, D, E, F, G, H, I, J являются вспомогательными - они не присутствуют на входной и входных лентах.

\subsection{Кодирование входной ленты}
Перед запуском машины на входной ленте должны быть представлены уменьшаемое и вычитаемое (уменьшаемое >= вычитаемому). Числа длжны быть разделены символом ''-''. Головка машины указывает на последний символ представления вычитаемого.

\subsection{Кодирование выходной ленты}
Результатом работы машины при корректных входных данных является разность между введенными числами. Головка машины указывает на первый символ этого представления.

\section{Построение машины}

\section{Результат построения}

\section{Выводы}
\LaTeX\ удобен для создания отчётов, так как сам следит за нумерацией таблиц, рисунков, листингов и отсылок к ним (так, например, здесь всегда будет указан номер рисунка "sample" не зависимо от того, какой он (1,2 или другой) - это рисунок \ref{pic:pic_name}). Не менее важно что весь документ оформлен в едином стиле, а исходные материалы подключаются к отчёту, а не хранятся в нём. Всё это позволяет легко получить качественный отчёт без дополнительных трат на его офрмление.

Исключения, пожалуй, составляют таблицы, так как их значительно сложнее создавать кодом, нежели в графическом редакторе. Но здесь никто не запрещает использовать визуальные средства создания таблиц для \LaTeX\ .
\end{document}
