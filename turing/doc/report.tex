\include{settings}

\begin{document}	% начало документа

% Титульная страница
\begin{titlepage}	% начало титульной страницы

	\begin{center}		% выравнивание по центру

		\large Санкт-Петербургский политехнический университет Петра Великого\\
		\large Институт компьютерных наук и технологий \\
		\large Высшая школа интеллектуальных систем и суперкомпьютерных технологий\\[6cm]
		% название института, затем отступ 6см
		
		\huge Низкоуровневое программирование\\[0.5cm] % название работы, затем отступ 0,5см
		\large Отчет по лабораторной работе №1\\[0.1cm]
		\large Машина Тьюринга-Поста\\[5cm]

	\end{center}


	\begin{flushright} % выравнивание по правому краю
		\begin{minipage}{0.25\textwidth} % врезка в половину ширины текста
			\begin{flushleft} % выровнять её содержимое по левому краю

				\large\textbf{Работу выполнил:}\\
				\large Аникин Д.А.\\
				\large {Группа:} 3530901/90004\\
				
				\large \textbf{Преподаватель:}\\
				\large Алексюк А.О.

			\end{flushleft}
		\end{minipage}
	\end{flushright}
	
	\vfill % заполнить всё доступное ниже пространство

	\begin{center}
	\large Санкт-Петербург\\
	\large \the\year % вывести дату
	\end{center} % закончить выравнивание по центру

\end{titlepage} % конец титульной страницы

\vfill % заполнить всё доступное ниже пространство


% Содержание
% Содержание
\renewcommand\contentsname{\centerline{Содержание}}
\setcounter{tocdepth}{1}
\tableofcontents
\newpage



\section{Цель работы}
Построить машину Тьюринга-Поста, вполняющую вычитание чисел в десятичном коде (уменьшаемое >= вычитаемому). Выполнить моделирование ее работы в одном из свободно доступных симуляторов. 

\section{Правила кодирования}

\subsection{Алфавит машины}
Алфавит машины состоит из следующих символов: ''-'' (разделение уменьшаемого и вычитаемого); Символы 1, 2, 3, 4,5, 6, 7, 8, 9,  используются для записи чисел в десятичном коде. Символы A, B, C, D, E, F, G, H, I, J являются вспомогательными - они не присутствуют на входной и входных лентах.

\subsection{Кодирование входной ленты}
Перед запуском машины на входной ленте должны быть представлены уменьшаемое и вычитаемое (уменьшаемое >= вычитаемому). Числа длжны быть разделены символом ''-''. Головка машины указывает на последний символ представления вычитаемого.

\subsection{Кодирование выходной ленты}
Результатом работы машины при корректных входных данных является разность между введенными числами. Головка машины указывает на первый символ этого представления.

\section{Построение машины}

\section{Результат построения}

\section{Выводы}
\LaTeX\ удобен для создания отчётов, так как сам следит за нумерацией таблиц, рисунков, листингов и отсылок к ним (так, например, здесь всегда будет указан номер рисунка "sample" не зависимо от того, какой он (1,2 или другой) - это рисунок \ref{pic:pic_name}). Не менее важно что весь документ оформлен в едином стиле, а исходные материалы подключаются к отчёту, а не хранятся в нём. Всё это позволяет легко получить качественный отчёт без дополнительных трат на его офрмление.

Исключения, пожалуй, составляют таблицы, так как их значительно сложнее создавать кодом, нежели в графическом редакторе. Но здесь никто не запрещает использовать визуальные средства создания таблиц для \LaTeX\ .
\end{document}
